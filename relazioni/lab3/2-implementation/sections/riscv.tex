\section{RISC-V Processor}
The processor is implemented as a pipelined processor where the execution of an instruction is divided in the following
stages:
\begin{enumerate}
    \item Instruction Fetch
    \item Instruction Decode
    \item Execution
    \item Memory
    \item Write Back
\end{enumerate}

Each stage is independently described within its own component which implements all of the necessary functionality. The
pipeline - specifically, the pipeline registers - is described and managed in the top level entity:
this means that every stage is implemented as a combinational component\footnote{Of course, single registers may be
needed in various occasions, the clearest instance is the Program Counter. But it is the duty of the single stage to
not interfere with the pipeline and internally deal with the components to maintain a correct timing}. This allows for
clean descriptions of the stages since they do not have to deal with instancing and controlling the pipeline registers.

The processor is also endowed with two specific units that improve the overall performance: the
\textbf{Forwarding Unit} and the \textbf{Branch Prediction Unit}, which minimize the number of pipeline flushes and
stalls introduced during the execution of a thread.

To simplify the description of the pipeline, the processor and the top level entities of the various stages use
VHDL records, all described in a global package. This allows for simpler signal assignments as well as a more versatile
implementation: whenever a stage needs a new signal that was not accounted for, or any modification in general is
needed, it is sufficient to update the record and the signal assignments, but no entity or instance has to be modified.

The specific descriptions of each unit, together with their design choices, functionalities and general discussions
are listed in the following sections.
