\section{Reference Model}
To properly verify the functionality of the filter, it is necessary to have a reference model that also works with 8 bit fixed point numbers. In addition, this allows for an insightful comparison of the filter's behavior when it is working in floating point (in Matlab) versus fixed point.


Truly fixed point operations are modeled in a C program where coefficients and data are stored using the standard \texttt{int} data type. A scale factor equal to $2^{n_b-1}$ is implied throughout the execution.\\
The loss in precision arising from truncating the least significant bits at the output of every hardware multiplier is simulated in C using the right-shift operator \texttt{>>}.

As shown in \autoref{eqn:iir}, there is a subtraction among the operations. To simplify the hardware, it is easier to transform the operation in an addition by changing the sign of $a_1$, as in
\begin{align}
	w[n] &= x[n] + (- a_1) w[n-1]
\end{align}
To better mirror this behavior, some small changes were necessary in the source files: first of all, $a_1$ had to be sign-flipped, then, inside the \texttt{myfilter} function, this section
\begin{Verbatim}
for (i=0; i<N; i++) {
	fb -= (sw[i]*a[i]) >> (NB-1);    // Subtraction here
	ff += (sw[i]*b[i]) >> (NB-1);
}
\end{Verbatim}
was changed into this
\begin{Verbatim}
for (i=0; i<N; i++) {
	fb += (sw[i]*a[i]) >> (NB-1);    // Addition here (a[i] changed sign)
	ff += (sw[i]*b[i]) >> (NB-1);
}
\end{Verbatim}
To clarify the difference, suppose to have:
\begin{align*}
	\texttt{NB} &= 8 & \texttt{a[i]} &= \pm21 \\
	\texttt{fb} &= 0 & \texttt{sw[i]} &= 1
\end{align*}
and notice how the results differ depending on the chosen operation and the actual sign of $\texttt{a[i]}$:
\setlength{\columnseprule}{0.4pt}
\begin{multicols}{2}
	\noindent
	\begin{align*}
		0 &- (1 \times -21) \rshift (8 - 1) &= \\
		0 &- 11101011_2 \rshift 7 &= \\
		0 &- 11111111_2 &= \\
		1&
	\end{align*}
	\begin{align*}
		0 &+ (1 \times -(-21)) \rshift (8 - 1) &= \\
		0 &+ 00010101_2 \rshift 7 &= \\
		0 &+ 00000000_2 &= \\
		0&
	\end{align*}
\end{multicols}
