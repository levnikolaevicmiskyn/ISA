\section{Synthesis}
The final design has been synthesized. The maximum frequency analysis is reported in \autoref{tab:maxtime}. In a second run, the clock period has been set to $4T_{min} = 4.8\,\textrm{ns}$, with the data reported in \autoref{tab:normaltime}.
\begin{table}
	\parbox[t]{0.53\textwidth}{
	\centering
	
\begin{tabular}[t]{|lrr|}  
	\hline
	\textbf{Point}                                                  & \textbf{Incr}   &    \textbf{Path}\\\hline
clock CLK (rise edge)                                 &  0.00     &   0.00\\
clock network delay (ideal)                          &    0.00     &  0.00\\
comp\_dp/w1\_reg[0]/CK                     &    0.00     &   0.00 \\
comp\_dp/w1\_reg[0]/Q                          &  0.10   &     0.10\\ comp\_dp/comp\_m2/a[0]  &0.00    &    0.10\\
comp\_dp/comp\_m2/mult\_31/a[0]  & 0.00     &   0.10       \\
...& &\\
comp\_dp/m2out\_del\_reg[6]/D       &              0.01    &    0.92\\
data arrival time               &                 &                    0.92\\
\hline
clock CLK (rise edge)                        &            0.00     &   0.00\\
clock network delay (ideal)                    &          0.00      &  0.00\\
clock uncertainty                              &         -0.07   &    -0.07\\
comp\_dp/m2out\_del\_reg[6]/CK  &                   0.00    &   -0.07 \\
library setup time                &                      -0.03    &   -0.10\\
data required time                                           & &      -0.10\\
\hline
data required time                                          &  &       -0.10\\
data arrival time                                             &  &     -0.92\\
\hline
slack (VIOLATED)                                                &  &   -1.02\\\hline

\end{tabular}



	\caption{Excerpt of the timing report with constraint $T_{ck}=0$ (yellow stage included)}
	\label{tab:maxtime}
	}
\hfill
	\parbox[t]{0.53\textwidth}{
	\centering
	\begin{tabular}[t]{|lrr|}\hline
\textbf{Point}                                      &             \textbf{Incr }   &   \textbf{Path}\\\hline
  clock CLK (rise edge)                                 &  0.00    &   0.00\\
  clock network delay (ideal)                          &   0.00     &  0.00\\
  comp\_dp/w1\_reg[3]/CK                         &  0.00    &   0.00 \\
  comp\_dp/w1\_reg[3]/Q                         &  0.19    &   0.19 \\
  comp\_dp/comp\_m3/a{[3]}   				& 0.00     &  0.19 \\
  comp\_dp/comp\_m3/mult\_31/a{[3]}                          &  0.00    &   0.19 \\
  comp\_dp/comp\_m3/mult\_31/U171/ZN               & 0.08    &   0.27 \\
  ...& &\\
  comp\_dp/comp\_m3/mult\_31/U3/S                     &0.13    &   1.63 \\
  comp\_dp/comp\_m3/mult\_31/product{[12]}       &0.00 &      1.63 \\
  comp\_dp/comp\_m3/y{[6]}  &0.00     &  1.63 \\
  comp\_dp/U15/ZN                              &  0.04    &   1.66 \\
  comp\_dp/a3a\_reg[6]/D                          & 0.01    &   1.67 \\
  data arrival time                                      & &            1.67\\
\hline
  clock CLK (rise edge)                                &   4.80   &    4.80\\
  clock network delay (ideal)                          &   0.00   &    4.80\\
  clock uncertainty                                    &  -0.07   &    4.73\\
  comp\_dp/a3a\_reg{[6]}/CK                       &   0.00  &    4.73 \\
  library setup time                                   &  -0.03  &     4.70\\
  data required time                                   &          &    4.70\\
  \hline
  data required time                                    &         &    4.70\\
  data arrival time                                       &        &  -1.67\\
  \hline
  slack (MET)                                             &        &   3.02\\\hline
\end{tabular}
	\caption{Excerpt of the timing report with constraint $T_{ck}=4T_{min}$ (yellow stage included)}
	\label{tab:normaltime}
}

\end{table}

\subsection{Post-synth simulation}
Synopsys can produce a Verilog description of the netlist generated by the compiler. This allows to verify the correctness of the outcome using ModelSim. Moreover, by running a second simulation after synthesis, reliable data regarding the switching activity associated to every internal node can be obtained and exported to enable an accurate power consumption estimation using the \texttt{report\_power} command provided by Synopsys. The results regarding power consumption are reported in \autoref{tab:power}, where the simulation has been performed with a continuous stream of data (\texttt{VIN} always equal to one). It might be interesting to compare this to the case where the filter is operated at half the data rate (\texttt{VIN} enabled every other clock cycle), as seen in \autoref{tab:power_2}.
\begin{table}
	\centering
	\begin{tabular}{|lllll|}
\hline
\textbf{Power group} &               \textbf{Internal Power}    &     \textbf{Switching Power}    &      \textbf{Leakage Power}      &     \textbf{Total Power}\\\hline
io\_pad            & 0.00         &   0.00 &           0.0000         &   0.00 (0.00\%)\\
memory            & 0.00        &    0.00  &          0.0000        &    0.00 (0.00\%)\\
black\_box         & 0.00       &     0.00   &         0.0000       &     0.00 (0.00\%)\\
clock\_network     & 0.00      &      0.00   &        0.0000      &      0.00 (0.00\%)\\
register         & 92.11     &       4.63     &   8.11e+03     &     104.86 (43.75\%)\\
sequential       &  0.61    &        0.91     &    326.98    &        1.85 (0.77\%)\\
combinational    & 61.40   &        49.11      &  2.24e+04   &       132.98 (55.48\%)\\
\hline
Total         &   154.1317 uW     &   54.6529 uW   &  3.0914e+04 nW     &  239.6988 uW \\\hline
\end{tabular}
	\caption{Power report with constraint $T_{ck}=4T_{min}$ (yellow stage included)}
	\label{tab:power}
\end{table}