\section{Synthesis}
The final design has been synthesized. The maximum frequency analysis is reported in \autoref{tab:maxtime}. In a second run, the clock period has been set to $4T_{min} = 4.8\,\textrm{ns}$, with the data reported in \autoref{tab:normaltime}.
\begin{table}
	\parbox[t]{0.53\textwidth}{
	\centering
	\begin{tabular}[t]{|lrr|}  
	\hline
\textbf{Point}                                                  & \textbf{Incr}   &    \textbf{Path}\\\hline
  clock CLK (rise edge)                                  & 0.00   &    0.00\\
  clock network delay (ideal)                             &0.00   &    0.00\\
  comp\_dp/w1\_reg[1]/CK                         & 0.00   &    0.00\\
  comp\_dp/w1\_reg[1]/Q                             &0.16   &    0.16\\
  comp\_dp/comp\_m2/a[1]    &0.00   &    0.16\\
  {[...]} & &\\
  comp\_dp/m2out\_del\_reg[5]/D                    &0.01   &    1.08\\
  data arrival time                                      &     &       1.08\\
\hline
  clock CLK (rise edge)                                  & 0.00  &     0.00\\
  clock network delay (ideal)                           &  0.00   &    0.00\\
  clock uncertainty                                     & -0.07  &    -0.07\\
  comp\_dp/m2out\_del\_reg[5]/CK                 &   0.00   &   -0.07\\
  library setup time                                   &  -0.04   &   -0.11\\
  data required time                                   &            & -0.11\\
\hline
  data required time                                    & &            -0.11\\
  data arrival time                                     & &            -1.08\\
\hline
  slack (VIOLATED)                                         & &         -1.19\\\hline
\end{tabular}
	\caption{Excerpt of the timing report with constraint $T_{ck}=0$ (yellow stage included)}
	\label{tab:maxtime}
	}
\hfill
	\parbox[t]{0.53\textwidth}{
	\centering
	\begin{tabular}{|lll|}\hline
Point                                      &             Incr    &   Path\\\hline
  clock CLK (rise edge)                                 &  0.00    &   0.00\\
  clock network delay (ideal)                          &   0.00     &  0.00\\
  comp\_dp/w1\_reg[3]/CK (DFF\_X1)                         &  0.00    &   0.00 \\
  comp\_dp/w1\_reg[3]/Q (DFF\_X1)                          &  0.19    &   0.19 \\
  comp\_dp/comp\_m3/a{[3]}   				& 0.00     &  0.19 \\
  comp\_dp/comp\_m3/mult\_31/a{[3]}                          &  0.00    &   0.19 \\
  comp\_dp/comp\_m3/mult\_31/U171/ZN (INV\_X1)               & 0.08    &   0.27 \\
  ...& &\\
  comp\_dp/comp\_m3/mult\_31/U3/S (FA\_X1)                    &0.13    &   1.63 \\
  comp\_dp/comp\_m3/mult\_31/product{[12]}       &0.00 &      1.63 \\
  comp\_dp/comp\_m3/y{[6]}  &0.00     &  1.63 \\
  comp\_dp/U15/ZN (AND2\_X1)                              &  0.04    &   1.66 \\
  comp\_dp/a3a\_reg[6]/D (DFF\_X1)                          & 0.01    &   1.67 \\
  data arrival time                                      & &            1.67\\
\hline
  clock CLK (rise edge)                                &   4.80   &    4.80\\
  clock network delay (ideal)                          &   0.00   &    4.80\\
  clock uncertainty                                    &  -0.07   &    4.73\\
  comp\_dp/a3a\_reg{[6]}/CK (DFF\_X1)                       &   0.00  &    4.73 \\
  library setup time                                   &  -0.03  &     4.70\\
  data required time                                   &          &    4.70\\
  \hline
  data required time                                    &         &    4.70\\
  data arrival time                                       &        &  -1.67\\
  \hline
  slack (MET)                                             &        &   3.02\\\hline
\end{tabular}
	\caption{Excerpt of the timing report with constraint $T_{ck}=4T_{min}$ (yellow stage included)}
	\label{tab:normaltime}
}

\end{table}

\subsection{Post-synth simulation}
Synopsys can produce a Verilog description of the netlist generated by the compiler. This allows to verify the correctness of the outcome using ModelSim. Moreover, by running a second simulation after synthesis, reliable data regarding the switching activity associated to every internal node can be obtained and exported to enable an accurate power consumption estimation using the \texttt{report\_power} command provided by Synopsys. The results regarding power consumption are reported in \autoref{tab:power}, where the simulation has been performed with a continuous stream of data (\texttt{VIN} always equal to one). 
\begin{table}
	\centering
	\begin{tabular}{|lllll|}
\hline
\textbf{Power group} &               \textbf{Internal Power}    &     \textbf{Switching Power}    &      \textbf{Leakage Power}      &     \textbf{Total Power}\\\hline
io\_pad            & 0.00         &   0.00 &           0.0000         &   0.00 (0.00\%)\\
memory            & 0.00        &    0.00  &          0.0000        &    0.00 (0.00\%)\\
black\_box         & 0.00       &     0.00   &         0.0000       &     0.00 (0.00\%)\\
clock\_network     & 0.00      &      0.00   &        0.0000      &      0.00 (0.00\%)\\
register         & 92.11     &       4.63     &   8.11e+03     &     104.86 (43.75\%)\\
sequential       &  0.61    &        0.91     &    326.98    &        1.85 (0.77\%)\\
combinational    & 61.40   &        49.11      &  2.24e+04   &       132.98 (55.48\%)\\
\hline
Total         &   154.1317 uW     &   54.6529 uW   &  3.0914e+04 nW     &  239.6988 uW \\\hline
\end{tabular}
	\caption{Power report with constraint $T_{ck}=4T_{min}$ (yellow stage included)}
	\label{tab:power}
\end{table}

\subsection{Place and Route} This process was carried out by using an automated script derived while routing the standard architecture with the Innovus GUI. The commands issued by the interface could be found in a \texttt{.cmd} file. The results for this step are available in the directory named \texttt{innovus\_fast} and can be fully replicated by running \texttt{source place\_and\_route.do} in Innovus.
\paragraph{Global routing} The global routing phase issued a few warnings claiming that a few pins do not have a physical counterpart and thus cannot be routed: they correspond to the MSB and LSB of the coefficients which are resized to match the internal representation. Synopsys had already detected that those interface pins are not used internally and optimized them out with a warning. Since this mismatch between interface and internal representation is wanted in the design and given that the reports resulting from this step are acceptable, these warnings can be safely ignored.
Global routing phase returns the following information regarding the total length routed on each layer (upper layer with 0 length are omitted).\\
\texttt{
\#Total wire length = 5618 um.\\
\#Total half perimeter of net bounding box = 6170 um.\\
\#Total wire length on LAYER metal1 = 6 um.\\
\#Total wire length on LAYER metal2 = 2967 um.\\
\#Total wire length on LAYER metal3 = 2556 um.\\
\#Total wire length on LAYER metal4 = 88 um.\\
\#Total wire length on LAYER metal5 = 0 um.\\
...\\
\#Total number of vias = 3817\\
}


\paragraph{Timing analysis} An inspection of the slacks reported in \texttt{IIRFilter\_postRoute.slk} shows that they are all positive and the lowest one is $2.53\,\textrm{ns}$. The summary shows that there are no violations.

\paragraph{Connectivity verification} This check confirms that there are no violations or warnings.

\paragraph{Gate count} The summary of total area and gate/cell count is summarized in \autoref{tab:gateCount}
\begin{table}
	\centering
\begin{tabular}{|l|l|}
	\hline
Gates &1962\\\hline
 Cells &     709\\\hline
  Area &    1565.7 $\mu\textrm{m}^2$\\\hline
\end{tabular}
\caption{Gate count summary}
\label{tab:gateCount}
\end{table}

\paragraph{Power analysis} Results are reported in \autoref{fig:innpwrvdd}, where power unit is mW.
\begin{table}[h]
	\centering
\begin{tabular}{|l|l|l|l|l|}
	\hline
\textbf{Group}                &           \textbf{Internal  Power} &  \textbf{Switching  Power}  &   \textbf{Leakage  Power}     & \textbf{Total  Power} \\
\hline
Sequential       &                 0.1723  &  0.003196  &  0.008441  &    0.1839 (64.39\%)  \\
Macro                       &           0    &       0       &    0      &     0  \\
IO                         &            0       &    0     &      0      &     0 \\
Combinational                &    0.04565   &  0.03347   &  0.02262   &   0.101 (35.61\%)   \\
Clock (Combinational)        &          0      &     0     &      0    &       0  \\
Clock (Sequential)             &        0     &      0      &     0     &      0 \\
\hline
Total                      &       0.218  &   0.03667   &  0.03106  &    0.2857 (100\%)  \\\hline
\hline
\textbf{CLK}                         &    0.1719   & 0.002668  & 0.008118    &  0.1827 (63.96\%)  \\\hline
\end{tabular}
\caption{Power report after place \& route: Group Power for Rail VDD and CLK}
\label{fig:innpwrvdd}
\end{table}

