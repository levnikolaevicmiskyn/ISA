\section{Vertools}
Vertools is a simple verification suite written in \textbf{Python} to conveniently carry all the above listed steps, automatically calling the necessary commands, managing logs and doing general cleanup operations before and after each run.\\
Given the variability of the verification process, Vertools was designed to give the user some level of customizability without the need of manually setting all the verification parameters at every run. For this reason, vertools makes use of \textbf{scoping} to give the user the possibility to set all the parameters in global or local configurations files and to manually override one or more of this parameters through the command line.

\subsection{Setting up the environment}
Vertools uses external Python packages. It is necessary to install them with Python's package manager, \textbf{pip}. To install the packages,
\begin{lstlisting}[language=bash]
    $ cd /path/to/vertools/
    $ python3 -m pip install -r requirements.txt
\end{lstlisting}
In shared machines, it could be better to install those packages for the current user only by adding the \texttt{--user} flag in the \texttt{python3 -m pip} command.

For Vertools to be able to run, it is necessary to add the path of the package folder to the \texttt{PYTHONPATH} environment variable with
\begin{lstlisting}[language=bash]
    $ export PYTHONPATH=${PYTHONPATH}:/path/to/vertools/
\end{lstlisting}

Finally, Vertools is a Python package, so it should be called with \texttt{python3 /path/to/vertools/}. The simplest solution to treat Vertools as its own command is to add an alias like this:
\begin{lstlisting}[language=bash]
    $ alias vertools="python3 /path/to/vertools/"
\end{lstlisting}

For simplicity, from now on it is assumed that the alias has been created and \texttt{vertools} will be treated as a command.

\subsection{Subcommands}
To improve the user's control over the operations, the tool's functionality was divided in subcommands. Each subcommand has its own purpose, carries its own functions and has its own flags and settings.

Beside the main \texttt{vertools verify} command, which runs the whole verification process, the user can call:

\begin{itemize}
    \item \texttt{vertools generate-inputs}: generate the input vectors choosing among various customizable waveforms.
    \item \texttt{vertools simulate}: set the simulation parameters (like clock period), call the provided simulation command (usually something like \texttt{source simulate.do}) save the logs and check if the expected output was generated.
    \item \texttt{vertools reference}: call the provided reference command (in this case, the C executable), store logs and check if the expected output was generated.
    \item \texttt{vertools compare}: compare the simulation and reference outputs, checking first and foremost if the lengths of the produced files are compatible, and then comparing them line by line.
    \item Other more specific commands which are not worth exploring here, but they can all be reviewed with \texttt{vertools --help}.
\end{itemize}

Each subcommand can have its own flags, options and even another level of subcommands (like \texttt{generate-inputs}). A specific help for each level of subcommand can be consulted by calling it with the \texttt{--help} flag.

\subsection{Configuration files}
It is possible to store the verification parameters in a configuration file. This way, it is not necessary to specify these parameters by hand each time. By default, Vertools looks for a file named \texttt{vertools.config} in the directory where the command is called. It is also possible to name the configuration file in other ways: this allows to choose between multiple configurations. In that case, the file should be specified with
\begin{lstlisting}[language=bash]
    $ vertools --config filename.config COMMAND [ARGUMENTS ...]
\end{lstlisting}
Configuration files support different \textbf{sections}, so that the same variable can have different values in different context (for example, \texttt{tstep} might be different between the input generation and the actual simulation step) and \textbf{interpolation}, meaning that it is possible to expand a variable in the assignment of another variable, bash-style.

\subsection{Program structure}
Vertools is a high level utility, meaning that it does not take part into the simulation or reference processes if not by managing and editing files and calling commands. This is because allowing a (newly written) software to handle delicate files like the VHDL files is dangerous.
The only instance where Vertools edits the content of a VHDL file is during the simulation phase, where it sets the clock period in the specified clock generator file with the value written in the configuration file.

Some of the most relevant modules and functionalities are discussed below.

\paragraph{cli.py} This is the module managing and setting the Command Line Interface. It uses the default \texttt{argparse} library which offers a helpful interface to handle command line arguments and flags (and to generate their help outputs) without the need of manually implementing a custom basic interpreter. As an example, the following lines of code generate the top level command (\texttt{vertools}) and its first subcommand (\texttt{generate-inputs}|\texttt{simulate}|...).
\begin{lstlisting}[language=Python, keywords={None}]
vertools = argparse.ArgumentParser(
    formatter_class=argparse.ArgumentDefaultsHelpFormatter,
    description='High level simulation tool for digital circuit verification'
)
vertools.add_argument(
    '--config',
    help='configuration file',
    metavar='FILE',
    dest='local_config',
    default=None
)
subparsers = vertools.add_subparsers(
    title='command',
    description='vertools command'
)
\end{lstlisting}

\paragraph{commands.py} In this module are implemented the actual command functionalities. All commands are implemented as classes inheriting from the custom \texttt{CommandAPI} class, which exposes three methods - \texttt{setup}, \texttt{run}, \texttt{exit}. These methods implement the respective steps of the execution of a command and, by default, are all called in sequence. This division can be helpful, for instance, when multiple commands share the same setup steps, allowing the user to define them only once.
\begin{lstlisting}[language=Python]
class CommandAPI:
"""Program command base class
Attributes:
    args (List): command arguments
    context (vertools.Context): contextualized parameters
    verbose (bool): flag to allow or block the command's output
    data (dict): custom data shared between the command phases
"""
def __init__(self, args, context, verbose=True):
    self.args = args
    self.context = context
    self.verbose = verbose
    self.data = {}

def setup(self):
    """Initialize files, context, variables, ...
    Returns:
        bool: True if command needs to continue, False to abort
    """
    return True

def run(self):
    """Run core actions"""
    pass

def exit(self):
    """Post execution actions"""
    pass

def output(self, output_func=output.update, *args, **kwargs):
    """Generate an output through a generic function only if command is set to verbose
    Args:
        output_func (function): output function (usually print or output.update)
        *args: output_func arguments
        **kwargs: output_func keyword arguments
    """
    if self.verbose is True:
        return output_func(*args, **kwargs)

def __call__(self):
    """Run the command by calling the setup, run and exit methods in order.
    If the setup method returns false, the command aborts.
    """
    setup_status = self.setup()
    if setup_status is False:
        return
    self.run()
    self.exit()
\end{lstlisting}

\paragraph{system.py} This module provides some wrapper functions to build an interface with the system. For example, the \texttt{run\_bash} function, which is used to call the simulation and reference commands, does all the necessary setup to run an external bash command and to redirect its output (either to logs or \texttt{/dev/null} when they are disabled).\\
It can receive a list of strings containing the bash commands which will all be executed in the same shell, which is useful for when some variables should be retained.

\begin{lstlisting}[language=Python, morekeywords={None}]
def run_bash(commands, **kwargs):
    """Run a bash command
    Args:
        commands (Union[str, List[str]]): command (or list of commands) to be executed in the same shell
        **kwargs: arbitrary keyword arguments
    Returns:
       subprocess.CompletedProcess
    """
    stdout = kwargs.get('stdout', None)
    stderr = kwargs.get('stderr', None)
    if stdout is False:
        stdout = subprocess.DEVNULL
    if stderr is False:
        stderr = subprocess.DEVNULL
    if isinstance(commands, list):
        command = ' && '.join(commands)
    else:
        command = commands
    status = subprocess.run(command, shell=True, stdout=stdout, stderr=stderr)
    return status
\end{lstlisting}

\paragraph{context.py} Module responsible for managing the scoping and context functionalities, which allow the user to work with configuration files and override some of the stored parameters through the command line. There are two fundamental classes, \texttt{Scope} and \texttt{Context}, which manage an arbitrarily deep parameter lookup hierarchy, maintaining the useful configuration files concept of sections.

\paragraph{waveforms.py} Module to generate the input data with various waveforms through the \texttt{scipy} and \texttt{numpy} libraries.
