\section{Vertools}
\label{sec:vertools}
Vertools is a simple verification tool written in \textbf{Python} used to aid the verification process. It automatically calls the necessary commands, checks the outputs, manages logs and does general cleanup operations before and after each run.

Vertools is an high level utility, meaning that it is not intrusive: it avoids, whenever possible, to edit any of the files used by the other programs or to perform similar operations. The only instance where Vertools edits the content of a VHDL file is during the simulation phase, where it sets the clock period in a specified clock generator. \\
This is not only because the verification process is logically supposed to not change the sources, but also because letting a (newly written) software fiddle with such sensitive files is dangerous. This leaves some operations which could, in theory, be implemented with an additional parameter to the hands of the user. For example, if multiple architectures are defined for the same component, it is up to the user to compile the right file or to manually edit the compilation command called by Vertools.

Before going into the details of its usage, it is necessary to correctly prepare the environment for its usage.

\subsection{Setting up the environment}
Vertools uses external Python packages which can be installed through Python's package manager, \textbf{pip}. To install them,
\begin{lstlisting}[language=bash]
    $ cd /path/to/vertools/
    $ python3 -m pip install -r requirements.txt
\end{lstlisting}
In shared machines, it could be better to install those packages for the current user only by adding the \texttt{--user} flag in the \texttt{python3 -m pip} command.

For Vertools to be able to run, it is necessary to add the path of the package folder to the \texttt{PYTHONPATH}.
\begin{lstlisting}[language=bash]
    $ export PYTHONPATH=${PYTHONPATH}:/path/to/vertools/
\end{lstlisting}

Finally, Vertools is a Python package, so it should be called with \texttt{python3 /path/to/vertools/}. It can be convenient to create an alias:
\begin{lstlisting}[language=bash]
    $ alias vertools="python3 /path/to/vertools/"
\end{lstlisting}

For simplicity, from now on it is assumed that the alias has been created and \texttt{vertools} will be treated as a command of its own.

\subsection{Subcommands}
To improve the granularity of the operations, the tool's functionality is divided in various levels of subcommands. Beside the main \texttt{vertools verify} command, which runs the whole verification process (simulation, reference, comparison), the user can call:

\begin{itemize}
    \item \texttt{vertools generate-inputs}: generate the input vectors choosing among various customizable waveforms.
    \item \texttt{vertools simulate}: set the simulation parameters (like clock period), call the provided simulation command (usually something like \texttt{source simulate.do}) save the logs and check if the expected output was generated.
    \item \texttt{vertools reference}: call the provided reference command (in this case, the C executable), store logs and check if the expected output was generated.
    \item \texttt{vertools compare}: compare the simulation and reference outputs, checking first and foremost if the lengths of the produced files are compatible, and then comparing them line by line.
    \item Other more specific commands which are not worth exploring here, but they can all be reviewed with \texttt{vertools --help}.
\end{itemize}

Each subcommand can have its own flags, parameters and even another level of subcommands. For instance, \texttt{generate-inputs} has a subcommand to specify which waveform should be created. Each waveform has a different number of parameters with different meanings. A specific help for each level of subcommand can be consulted by calling it with the \texttt{--help} flag, as in \texttt{vertools generate-inputs sine --help}, which will show the order in which to specify the sine wave parameters.

\subsection{Configuration files}
To grant some level of customizability and reusability, Vertools' operations are parametrized. It is possible to store those verification parameters in a configuration file: this way, it is not necessary to specify them by hand at each run.
Because Vertools works with the concept of \textit{scoping}, the user can override specific parameters through the appropriate command line options without actually editing the configuration file.

By default, Vertools looks for a file named \texttt{vertools.config} in the directory where the command is called. It is also possible to name the configuration file in other ways: this allows to choose between multiple configurations. In that case, the file should be specified with
\begin{lstlisting}[language=bash]
    $ vertools --config filename.config COMMAND [ARGUMENTS ...]
\end{lstlisting}
Configuration files support different \textbf{sections}, so that the same variable can have different values in different context (for example, \texttt{tstep} might be different between the input generation and the actual simulation step) and \textbf{interpolation}, meaning that it is possible to expand a variable in the assignment of another variable, bash-style.

\subsection{Program structure}
The program is composed of multiple modules, responsible to implement its various parts. Some of the modules can be customized to provide additional or ad-hoc functionalities.

\paragraph{cli.py} This is the module setting and managing the Command Line Interface. It uses the default \texttt{argparse} library which offers a helpful interface to handle command line arguments and flags and to generate their help menus without the need of manually implementing a custom basic interpreter. As an example, the following lines of code generate the top level command (\texttt{vertools}) and its first subcommand (\texttt{generate-inputs}/\texttt{simulate}/...).
\begin{lstlisting}[language=Python, keywords={None}]
vertools = argparse.ArgumentParser(
    formatter_class=argparse.ArgumentDefaultsHelpFormatter,
    description='High level simulation tool for digital circuit verification'
)
vertools.add_argument(
    '--config',
    help='configuration file',
    metavar='FILE',
    dest='local_config',
    default=None
)
subparsers = vertools.add_subparsers(
    title='command',
    description='vertools command'
)
\end{lstlisting}

\paragraph{commands.py} The actual command procedures are implemented here. All commands are implemented as classes inheriting from the custom \texttt{CommandAPI} class, which exposes three methods - \texttt{setup}, \texttt{run}, \texttt{exit}. These methods implement the respective steps of the execution of a command and, by default, are all called in sequence when the command is requested. This division can be helpful, for instance, when multiple commands share the same setup steps, allowing the user to define them only once. Furthermore, the setup function can also halt the execution of the command by returning \texttt{False} when something goes wrong.
\begin{lstlisting}[language=Python]
class CommandAPI:
"""Program command base class
Attributes:
    args (List): command arguments
    context (vertools.Context): contextualized parameters
    verbose (bool): flag to allow or block the command's output
    data (dict): custom data shared between the command phases
"""
def __init__(self, args, context, verbose=True):
    self.args = args
    self.context = context
    self.verbose = verbose
    self.data = {}

def setup(self):
    """Initialize files, context, variables, ...
    Returns:
        bool: True if command needs to continue, False to abort
    """
    return True

def run(self):
    """Run core actions"""
    pass

def exit(self):
    """Post execution actions"""
    pass

def output(self, output_func=output.update, *args, **kwargs):
    """Generate an output through a generic function only if command is set to verbose
    Args:
        output_func (function): output function (usually print or output.update)
        *args: output_func arguments
        **kwargs: output_func keyword arguments
    """
    if self.verbose is True:
        return output_func(*args, **kwargs)

def __call__(self):
    """Run the command by calling the setup, run and exit methods in order.
    If the setup method returns false, the command aborts.
    """
    setup_status = self.setup()
    if setup_status is False:
        return
    self.run()
    self.exit()
\end{lstlisting}

\paragraph{system.py} This module provides some wrapper functions to build an interface with the system. For example, the \texttt{run\_bash} function (which is the one used to call the simulation and reference commands) handles all the necessary steps to run an external bash command in a subshell and to redirect its output either to logs or \texttt{/dev/null}, when they are disabled.\\
It can receive a list of strings containing the bash commands which will all be executed in the same shell, which is useful for when some environment variables are shared.

\begin{lstlisting}[language=Python, morekeywords={None}]
def run_bash(commands, **kwargs):
    """Run a bash command
    Args:
        commands (Union[str, List[str]]): command (or list of commands) to be executed in the same shell
        **kwargs: arbitrary keyword arguments
    Returns:
       subprocess.CompletedProcess
    """
    stdout = kwargs.get('stdout', None)
    stderr = kwargs.get('stderr', None)
    if stdout is False:
        stdout = subprocess.DEVNULL
    if stderr is False:
        stderr = subprocess.DEVNULL
    if isinstance(commands, list):
        command = ' && '.join(commands)
    else:
        command = commands
    status = subprocess.run(command, shell=True, stdout=stdout, stderr=stderr)
    return status
\end{lstlisting}

\paragraph{context.py} Module responsible for managing the scoping and context functionalities, which allow the user to work with configuration files and override some of the stored parameters through the command line. There are two fundamental classes, \texttt{Scope} and \texttt{Context}, which manage an arbitrarily deep parameter lookup hierarchy, maintaining the useful configuration files concept of sections.

\paragraph{waveforms.py} Module to generate the input data with various waveforms through the \texttt{scipy} and \texttt{numpy} libraries.
