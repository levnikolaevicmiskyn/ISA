\section{Methodology}
The task of verifying a design using a logical simulator such as ModelSim can be broken down into four distinct phases:
\begin{enumerate}
    \item Generation of suitable input vectors
    \item A number of logical simulations where the \textit{device-under-test} is operated within a testbench environment, providing it with the generated inputs and recording its outputs
    \item A number of reference software runs, where the ideal behavior and the reference results are generated
    \item Analysis and comparison of the output data against the reference results
\end{enumerate}

\paragraph{Targets} In order to plan the verification flow, all aspects of the design that need to be tested must be identified. In the case of a data processing unit like the one we are considering, the correctness of the \textbf{output data} stream must be checked along with the \textbf{timing} of every control and status signal.

\paragraph{VHDL testbench structure} The testbench is made up of the following non-synthesizable components:
\begin{itemize}
	\item Clock generator (\texttt{clkGen.vhd}): it generates a clock signal whose period is parametrized by a constant written externally in a dedicated package (\texttt{work.simconsts}).
	\item Data generator (\texttt{dataGen.vhd}): it fetches the input samples from a text file to feed the DUT at regular intervals, with a frequency submultiple of $f_{clk}$.
	\item Monitor (\texttt{dataSink.vhd}): it collects output samples in a text file for further analysis and verifies that the status signal \texttt{VOUT} is asserted with the right timing. This is done by probing the control signal \texttt{VIN} and ensuring that \texttt{VOUT} is issued after a latency period (in clock cycles) given as a constant and dependent on the internal architecture. If any unexpecyed behavior is detected, a warning message appears on the output log.
\end{itemize}

\paragraph{Data generation and analysis} For the creation of the input samples file we resorted to an automated script that provides several options to generate waveforms such as the impulse, the unit step, a sinewave or a chirp signal whose parameters (frequency and amplitude) are given as arguments. Once the input data is stored in a text file, suitable commands are issued to run a simulation in ModelSim. Finally, the output of the monitor component is checked against the stream produced by the reference C model.
\todo{Complete the verification flow}
