\chapter{Universal Verification Methodology}
\section{Introduction}
The UVM is an open source verification standard consisting in a set of classes that extend the SystemVerilog language with advanced verification capabilities. A UVM testbench is made up of reusable components that are often an extension of the base classes already available and fit into a standardized testbench architecture. The object-oriented approach brings the verification task at an higher level of abstraction, which makes UVM a flexible and efficient environment.

The general organization separates the verification domain from the DUT and its interface. Only the interface that closely interacts with the DUT needs to specify communication at the signal-level, while the remaining part that carries out most of the verification can operate at the transaction level (Transaction Level Model).

\paragraph{TLM} Transactions are objects that model all the abstract features of communication between two objects. Interfaces for transaction-level communication are used in UVM to isolate the internal implementation of each component so that different components with the same interface can be swapped easily.
The elementary chunk of data exchanged in a transaction is an object of a custom class derived from \texttt{uvm\_sequence\_item}. This object includes all the variables, constraints and methods required to fully specify how data is structured and delivered in a given transaction.

Instances of such sequence items exist in the testbench under analysis in two classes: \texttt{packet\_in} and \texttt{packet\_out}. They model the data generated by the sequencer and the packets collected at the DUT's output for checking/scoreboarding. In particular, \texttt{packet\_in} will deserve more attention since it describes the input sequence item as a couple of random integers $A$ and $B$ along with constraints that represent the properties that a valid input sequence must have.

Exchange of data happens, at the most simple level, between a producer component that \textit{puts} data on a port and a consumer with an associated export that \textit{gets} the transaction. The producer is the one that generates a transaction and sends it to its port using the \texttt{put} method. On the other hand, the consumer is expected to give an implementation of the \texttt{put} method, which is called whenever the producer wants to put data to its consumer.

The converse operation on the consumer's side is to ask for a transaction from the producer. This is accomplished with the \texttt{get} method, called by the consumer and implemented in the producer.
For example, the \texttt{refmod} component contains both a get port and a put port objects, which allow to receive transactions from the agent and to send the result to the scoreboard.

\begin{lstlisting}[language = verilog, caption = Snippet from \textit{refmod.sv}]
packet_in tr_in;
packet_out tr_out;
uvm_get_port #(packet_in) in;
uvm_put_port #(packet_out) out;

virtual task run_phase(uvm_phase phase);
	[...]
	forever begin
		in.get(tr_in); // Get transaction from agent
		[...]
		out.put(tr_out); // Put transaction to the scoreboard
	end
endtask: run_phase
\end{lstlisting}

In this case the transactions are \textit{non-blocking}, they return instantaneously in the same delta cycle in which they are invoked.

\paragraph{Cycle} Every UVM object undergoes three distinct phases during a simulation run: all classes are required to give an implementation for these in the form of methods.
The \textbf{build phase} is where objects are created by means of the \texttt{create} method defined in the \texttt{uvm\_registry\_component} wrapper class. This way of creating objects is prescribed by the factory pattern, a paradigm drawn from object-oriented programming that is here employed to allow customization of objects through a configuration database rather than direct modification of the testbench code. Macros are used in every class to \textit{register} the object to the factory (as in line 2 of \ref{driver}). For instance, \texttt{uvm\_component\_utils} and \texttt{uvm\_object\_utils} are the macros for registering a new class type, respectively for classes derived from \texttt{uvm\_component} or \texttt{uvm\_object} classes \cite{sistenix} \todo{Add .bib file}.

In the following snippet, the defult constructor simply calls its conterpart in the parent with default parameters to create an object whose default properties will be overridden later by means of the create method.

\begin{lstlisting}[language=verilog, label=driver]
class driver extends uvm_driver #(packet_in);
	`uvm_component_utils(driver) // Registration to the factory
	[...]
	// Constructor
	function new(string name = "driver", uvm_component parent = null);
		super.new(name, parent);
	endfunction

	// Build phase
	virtual function void build_phase(uvm_phase phase);
		super.build_phase(phase);
		assert(uvm_config_db#(input_vif)::get(this, "", "vif", vif));
	endfunction
\end{lstlisting}

The \textbf{connect phase} follows, where properly defined function create connection among testbench components by calling the \texttt{connect} method. As an example, the following snippet
shows the driver-sequencer connection and the link between the monitor and an \texttt{uvm\_analysis\_port} object.

\begin{lstlisting}[language=verilog, caption=Creation phase from the agent]
 virtual function void connect_phase(uvm_phase phase);
		super.connect_phase(phase);
		mon.item_collected_port.connect(item_collected_port); // Monitor connection
		drv.seq_item_port.connect(sqr.seq_item_export);       // Driver connection with sequencer
 endfunction
	\end{lstlisting}

Finally, the testbench is executed in the \textbf{run phase} and information are displayed in  the \textbf{report phase}.
\section{Architecture}
The testbench infrastructure used in this lab is freely provided by \textit{sistenix.com}. In the following, we report a brief overview of how different tasks are partitioned across components taking this particular implementation as a reference. This is still general, since the architecture is the one specified in the UVM standard \cite{uvm_book}, and can be expected to be essentially the same for all compliant implementations.

At the highest hierarchical level, verification tasks are performed in the \texttt{env} (\textit{environment}) entity, which instantiates the agents and the scoreboard.

\paragraph{Agent} The UVM standard defines the agent as the combination of a sequencer that produces the stimuli, a driver that communicates those stimuli to the DUT's interface and a monitor
to gather the output data flow. This testbench includes a dedicated agent that operates in \textit{active} mode by driving the DUT's input interface. In particular, the \textit{sequencer} implements constrained random input generation at the transaction level, whose packet flow is then translated to bit-level signals by the \textit{driver}. These inputs must be forwarded to the scoreboard, this is accomplished by the \textit{monitor} that takes the signals at the driver's output back to the transaction level and puts them to the \textit{refmod}.

A separate \textit{passive} agent collects the output data and it is made up of a driver and a monitor directly connected to the DUT's output interface.

\paragraph{Scoreboard} This component is in charge of verifying that the output results match those of a "golden" reference model for a given input. The behavioral description in the \textit{refmod} is an high level representation of the task that the DUT is supposed to perform: the design under test is correctly verified if it reproduces the refmod under all conditions.


\section{Constrained random input}
The goal of verification is, of course, to ensure the design correctness. Ideally, this should be checked throughout the whole space of possible input patterns. However, the combinations of legal input sequences are almost always too many to be tested in simulation, because it would simply take too long to complete. Hence, any practically feasible verification goal would consist in checking a small subset of all cases that is still representative of the whole set. Therefore, the target of a verification engineer is full coverage, which is a percentage that measures which fraction of all the interesting operating conditions have been verified. Complete coverage means that the design has been tested for all the conditions that are needed to be sure that it will work in every condition, including corner cases.

Driving the DUT with a completely random input generator is not efficient since it might fail in exploring corner cases or even to give a fair representation on the input domain, with similar cases that might be tested multiple times needlessly. This hints that the sequence generation should directed by the engineer with suitable constraints. Given the importance of constrained random verification, there are specific tools in UVM that address this need. First of all, the keyword rand allows us to easily define a random variable, to which constraints may be attached to specify its range (the set of values that it can take) and its statistical distribution. For instance, the following code defines a random integer A in the interval [10,100] and a second variable B whose range, at any given time, is dependent on A.
\begin{lstlisting}[language=verilog, caption= Constrained random sequencer from \textit{packet\_in.sv}]
	rand integer A;
	rand integer B;

	constraint myrangeA {A inside {[100:1000]};}
	constraint myrangeB {B < 10*A; solve A before B;}
\end{lstlisting}
% This part is worth a closer look as it implements an handshake-based(?) communication protocol with the DUT.
