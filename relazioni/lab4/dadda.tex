\chapter{Dadda Tree Multiplier}
Given the described simulation environment, a simulation of the Dadda Tree Multiplier from Lab 2 was executed.
Since the top level component of the multiplier is described in VHDL, it was no longer sufficient to include its design
in \texttt{top.sv}: it is necessary to explicitly create an instance:
\begin{lstlisting}[
    language = verilog,
    caption = Component instantiation in \texttt{DUT.sv}]
MBE #(.N(32)) mbe_under_test(.m_and(in_inter.A),.m_ier(in_inter.B),.z(out_inter.data));
\end{lstlisting}

The following step is to adjust the data interface.

\begin{lstlisting}[
    language = verilog,
    caption = DUT interface, 
    label = code:packet_in]
interface dut_if(input clk, rst);
    logic [31:0] A, B;
    logic [63:0] data;
    logic valid, ready;

    modport port_in (input clk, rst, A, B, valid, output ready);
    modport port_out (input clk, rst, output valid, data, ready);
endinterface
\end{lstlisting}
The only needed change in the input interface in this case was to increase the output data width from 32 bits to 64 bits according to the
component's output itself, as shown in \ref{code:packet_in}.

As for the output interface, the variable \texttt{data} that stores the result from the \texttt{refmod} must be a \texttt{longint} to accommodate a 64-bit result. 

Finally, it is possible to apply constraints on the randomly generated numbers. Being our DUT an unsigned multiplier it is
best to constrain the inputs to be non-negative in order to prevent confusion with the reference model and to adapt
the model itself by simply changing the addition into a multiplication.

\begin{lstlisting}[
    language = verilog,
    caption = Input constraints]
class packet_in extends uvm_sequence_item;
    rand integer A;
    rand integer B;

	constraint myrangeA {A >= 0;}
	constraint myrangeB {B >= 0;}

    `uvm_object_utils_begin(packet_in)
        `uvm_field_int(A, UVM_ALL_ON|UVM_HEX)
        `uvm_field_int(B, UVM_ALL_ON|UVM_HEX)
    `uvm_object_utils_end

    function new(string name="packet_in");
        super.new(name);
    endfunction: new
endclass: packet_in

\end{lstlisting}

Running the simulation with this setup shows the expected 101/101 matches.
